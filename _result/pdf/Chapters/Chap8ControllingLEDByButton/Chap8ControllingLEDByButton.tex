% -*- mode: latex; -*- mustache tags:  
\documentclass[10pt,twoside,english]{_support/latex/sbabook/sbabook}
\let\wholebook=\relax

\usepackage{import}
\subimport{_support/latex/}{common.tex}

%=================================================================
% Debug packages for page layout and overfull lines
% Remove the showtrims document option before printing
\ifshowtrims
  \usepackage{showframe}
  \usepackage[color=magenta,width=5mm]{_support/latex/overcolored}
\fi


% =================================================================
\title{A PharoThings Tutorial}
\author{Allex Oliveira}
\series{Square Bracket tutorials}

\hypersetup{
  pdftitle = {A PharoThings Tutorial},
  pdfauthor = {Allex Oliveira},
  pdfkeywords = {IoT, Raspberry, PharoThings, Pharo}
}


% =================================================================
\begin{document}

% Title page and colophon on verso
\maketitle
\pagestyle{titlingpage}
\thispagestyle{titlingpage} % \pagestyle does not work on the first one…

\cleartoverso
{\small

  Copyright 2017 by Allex Oliveira.

  The contents of this book are protected under the Creative Commons
  Attribution-ShareAlike 3.0 Unported license.

  You are \textbf{free}:
  \begin{itemize}
  \item to \textbf{Share}: to copy, distribute and transmit the work,
  \item to \textbf{Remix}: to adapt the work,
  \end{itemize}

  Under the following conditions:
  \begin{description}
  \item[Attribution.] You must attribute the work in the manner specified by the
    author or licensor (but not in any way that suggests that they endorse you
    or your use of the work).
  \item[Share Alike.] If you alter, transform, or build upon this work, you may
    distribute the resulting work only under the same, similar or a compatible
    license.
  \end{description}

  For any reuse or distribution, you must make clear to others the
  license terms of this work. The best way to do this is with a link to
  this web page: \\
  \url{http://creativecommons.org/licenses/by-sa/3.0/}

  Any of the above conditions can be waived if you get permission from
  the copyright holder. Nothing in this license impairs or restricts the
  author's moral rights.

  \begin{center}
    \includegraphics[width=0.2\textwidth]{_support/latex/sbabook/CreativeCommons-BY-SA.pdf}
  \end{center}

  Your fair dealing and other rights are in no way affected by the
  above. This is a human-readable summary of the Legal Code (the full
  license): \\
  \url{http://creativecommons.org/licenses/by-sa/3.0/legalcode}

  \vfill

  % Publication info would go here (publisher, ISBN, cover design…)
  Layout and typography based on the \textcode{sbabook} \LaTeX{} class by Damien
  Pollet.
}


\frontmatter
\pagestyle{plain}

\tableofcontents*
\clearpage\listoffigures

\mainmatter

\chapter{Lesson 7 - Controlling LED by Button}
In the previous lessons, we learned how to control the GPIOs putting them in mode OUT. This means send power to wire connected in on specific GPIO. Now we will put the GPIO in mode IN, to read the pin state. This means that our application can know when a button is pressed. Let’s create a shortcode on the remote inspector to turn On one LED each time we press the button.
\section{What do we need?}
We are using the set of the first lesson, but let's use 8 LEDs and 8 resistors and some more jumper wires.
\subsection{Components}
\begin{itemize}
\item 1 Raspberry Pi connected to your network (wired or wireless)
\item 1 Breadboard
\item 1 LED
\item 1 Resistors 330ohms
\item 1 Push button
\item Jumper wires
\end{itemize}
\section{Experimental theory}
The Raspberry Pi GPIOs can be set OUT or IN mode. When we set them to IN, the selected pin can have change the state from 0 to 1 or vice-versa when we connect it in the 3.3V or ground PIN.

In this lesson, we will use the pull-up mode with an internal resistor. This means that we will use the 3.3V pin to change the GPIO value. If we used the pull-down mode, we would use the Ground Pin to change the GPIO value. To know more, you can access the Wiring Pi website, http://wiringpi.com/reference/core-functions.

In this scenario, we will put a push button between the 3.3V pin and GPIO pin, to control when to send energy to GPIO.
\section{Experimental procedure}
Now we will build the circuit. This circuit consists of an LED, a resistor to limit current and a push button, to control when to send power to GPIO IN.  

\begin{itemize}
\item Connect the Ground PIN from Raspberry in the breadboard blue rail (-). In this experiment we will use the PIN6 (Ground);
\item Then connect the resistor from the same row on the breadboard to a column on the breadboard, as shown below;
\item Now push the LED legs into the breadboard, with the long leg (with the kink) on the right;
\item And insert a jumper wire connecting the right column and the PIN7 (GPIO7);
\item Connect the 3.3V PIN in the red rail (+);
\item Then insert the push button on the breadboard and connect one leg on PIN11 (GPIO0);
\item And another button leg on the red rail (+). Pay attention to the position of the button!
\end{itemize}

The figure below shows how the electric connection is made:
\section{Connecting remotely}
Through your local Pharo image, let’s connect in the Pharo image by running on Raspberry, enable the auto-refresh feature of the inspector, and open the inspector.
Run this code in your local playground:

\begin{displaycode}{plain}
remotePharo := TlpRemoteIDE connectTo: (TCPAddress ip: #[193 51 236 212] port: 40423)
GTInspector enableStepRefresh.
remoteBoard := remotePharo evaluate: [ RpiBoard3B current].
remoteBoard inspect.
\end{displaycode}
\section{Experimental code}
In your inspect window (Inspector on a PotRemoteBoard), let’s create the instances of the LED and button.

To instantiate the LED, let’s do how we did previous, putting it in beDigitalOutput mode:

\begin{displaycode}{plain}
led := gpio7
led beDigitalOutput.
\end{displaycode}

To instantiate the button let’s do the same, but putting it in beDigitalInput and setting it to enablePullUpResister mode:

\begin{displaycode}{plain}
button := gpio0.
button beDigitalInput.
button enablePullUpResister.
\end{displaycode}

This means each time that we connect the 3.3V on this GPIO (pushing the button), the state will be changed to 1, and back to 0 after release the button.

You ask the value of the button using the value method. Do this test running this line with the button pressed and after with the button released. This test is good for you to check if your button is working correctly:

\begin{displaycode}{plain}
button value.
\end{displaycode}

Now let’s create a process to check when the button is pressed and send this value to led object:

\begin{displaycode}{plain}
[ [100 milliSeconds wait. 
	led value: (button value=1) asBit
		] repeat	
	 ] forkNamed: 'button process'.
\end{displaycode}

Your LED will turn On when the button is pressed!
\section{Terminating the process}
As we saw in the Blinking LED lesson, you can finish this process remotely, case you don’t want to wait it finish. To do this, call the Remote Process Browser:

\begin{displaycode}{plain}
 remotePharo openProcessBrowser.
\end{displaycode}

Search the FlowingProcess and terminate it:

\begin{itemize}
\item selecting the process and using the shortcut “Cmd + T”;
\item selecting the process and using the button Terminate;
\item or right-click and select Terminate.
\end{itemize}
\section{In the next lesson}
In this lesson, you learned how to configure a push button, take the value of the button and send it to LED value, doing the LED turn On or Off using the button.

Next lesson we will start to use the sensors using I2C protocol. We will see the BME280 (temperature, humidity, and air pressure), ADXL (accelerometer X, Y, Z) and MCP9808 (temperature).


% lulu requires an empty page at the end. That's why I'm using
% \backmatter here.
\backmatter

% Index would go here

\end{document}
