% -*- mode: latex; -*- mustache tags:  
\documentclass[10pt,twoside,english]{_support/latex/sbabook/sbabook}
\let\wholebook=\relax

\usepackage{import}
\subimport{_support/latex/}{common.tex}

%=================================================================
% Debug packages for page layout and overfull lines
% Remove the showtrims document option before printing
\ifshowtrims
  \usepackage{showframe}
  \usepackage[color=magenta,width=5mm]{_support/latex/overcolored}
\fi


% =================================================================
\title{A PharoThings Tutorial}
\author{Allex Oliveira}
\series{Square Bracket tutorials}

\hypersetup{
  pdftitle = {A PharoThings Tutorial},
  pdfauthor = {Allex Oliveira},
  pdfkeywords = {IoT, Raspberry, PharoThings, Pharo}
}


% =================================================================
\begin{document}

% Title page and colophon on verso
\maketitle
\pagestyle{titlingpage}
\thispagestyle{titlingpage} % \pagestyle does not work on the first one…

\cleartoverso
{\small

  Copyright 2017 by Allex Oliveira.

  The contents of this book are protected under the Creative Commons
  Attribution-ShareAlike 3.0 Unported license.

  You are \textbf{free}:
  \begin{itemize}
  \item to \textbf{Share}: to copy, distribute and transmit the work,
  \item to \textbf{Remix}: to adapt the work,
  \end{itemize}

  Under the following conditions:
  \begin{description}
  \item[Attribution.] You must attribute the work in the manner specified by the
    author or licensor (but not in any way that suggests that they endorse you
    or your use of the work).
  \item[Share Alike.] If you alter, transform, or build upon this work, you may
    distribute the resulting work only under the same, similar or a compatible
    license.
  \end{description}

  For any reuse or distribution, you must make clear to others the
  license terms of this work. The best way to do this is with a link to
  this web page: \\
  \url{http://creativecommons.org/licenses/by-sa/3.0/}

  Any of the above conditions can be waived if you get permission from
  the copyright holder. Nothing in this license impairs or restricts the
  author's moral rights.

  \begin{center}
    \includegraphics[width=0.2\textwidth]{_support/latex/sbabook/CreativeCommons-BY-SA.pdf}
  \end{center}

  Your fair dealing and other rights are in no way affected by the
  above. This is a human-readable summary of the Legal Code (the full
  license): \\
  \url{http://creativecommons.org/licenses/by-sa/3.0/legalcode}

  \vfill

  % Publication info would go here (publisher, ISBN, cover design…)
  Layout and typography based on the \textcode{sbabook} \LaTeX{} class by Damien
  Pollet.
}


\frontmatter
\pagestyle{plain}

\tableofcontents*
\clearpage\listoffigures

\mainmatter

\chapter{Lesson 5 - LED Flowing Lights using OOP}
Now we can play with the LEDs, turn them on, off, blink it and manipulate many at the same time. Let’s use object-oriented programming, OOP to create methods and classes, to build a simple program, to control the LEDs flow like as we want.

\begin{displaycode}{plain}
leds := Flowing new.
leds times: 2 delay: 0.1 direction: 'lrl'.
leds flowStart.
leds flowStop.
leds turnOn.
leds turnOff.
\end{displaycode}

\begin{displaycode}{plain}
Object subclass: #Flowing
	instanceVariableNames: 'ledArray flowProcess flowDirection toggleDelay timesRepeat'
	classVariableNames: ''
	package: 'PharoThings-Lessons'
\end{displaycode}

\begin{displaycode}{plain}
initialize
	ledArray := { 
	(PotGPIOPin id: 17 number: 0).
	(PotGPIOPin id: 18 number: 1).
	(PotGPIOPin id: 27 number: 2).
	(PotGPIOPin id: 22 number: 3).
	(PotGPIOPin id: 23 number: 4).
	(PotGPIOPin id: 24 number: 5).
	(PotGPIOPin id: 25 number: 6).
	(PotGPIOPin id: 4 number: 7)
	}.
   ledArray do: [ :item | item board: RpiBoardBRev2 current; beDigitalOutput; value:0 ].
	timesRepeat := 2.
	toggleDelay := 0.5.
	flowDirection := 'lr'.
\end{displaycode}

\begin{displaycode}{plain}
times: anInteger delay: aNumber direction: aString
	timesRepeat := anInteger.
	toggleDelay := aNumber.
	flowDirection := aString
\end{displaycode}

\begin{displaycode}{plain}
flowDirection
	^flowDirection
\end{displaycode}

\begin{displaycode}{plain}
flowTimesRepeat
	^timesRepeat
\end{displaycode}

\begin{displaycode}{plain}
toggleDelay
	 ^toggleDelay
\end{displaycode}

\begin{displaycode}{plain}
flowStart
	flowProcess := [ (self flowTimesRepeat) timesRepeat: [
          self action
      ] ] forkNamed: 'FlowingProcess'.
\end{displaycode}

\begin{displaycode}{plain}
flowStop
	flowProcess terminate
\end{displaycode}

\begin{displaycode}{plain}
action
	flowDirection = 'lr' ifTrue: [  ledArray do: self toggleLedArray ].
	flowDirection = 'rl' ifTrue: [  ledArray reverseDo: self toggleLedArray ].
	flowDirection = 'lrl' ifTrue: [  ledArray do: self toggleLedArray; reverseDo: self toggleLedArray ].
	flowDirection = 'rlr' ifTrue: [  ledArray reverseDo: self toggleLedArray; do: self toggleLedArray ]
\end{displaycode}

\begin{displaycode}{plain}
toggleLedArray
	^[ :item | item toggleDigitalValue. (Delay forSeconds: self toggleDelay) wait ]
\end{displaycode}

\begin{displaycode}{plain}
turnOn
	ledArray do: [ :item | item value:1 ].
\end{displaycode}

\begin{displaycode}{plain}
turnOff
	ledArray do: [ :item | item value:0 ].
\end{displaycode}


% lulu requires an empty page at the end. That's why I'm using
% \backmatter here.
\backmatter

% Index would go here

\end{document}
