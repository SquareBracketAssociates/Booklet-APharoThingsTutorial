% -*- mode: latex; -*- mustache tags:  
\documentclass[10pt,twoside,english]{_support/latex/sbabook/sbabook}
\let\wholebook=\relax

\usepackage{import}
\subimport{_support/latex/}{common.tex}

%=================================================================
% Debug packages for page layout and overfull lines
% Remove the showtrims document option before printing
\ifshowtrims
  \usepackage{showframe}
  \usepackage[color=magenta,width=5mm]{_support/latex/overcolored}
\fi


% =================================================================
\title{A PharoThings Tutorial}
\author{Allex Oliveira}
\series{Square Bracket tutorials}

\hypersetup{
  pdftitle = {A PharoThings Tutorial},
  pdfauthor = {Allex Oliveira},
  pdfkeywords = {IoT, Raspberry, PharoThings, Pharo}
}


% =================================================================
\begin{document}

% Title page and colophon on verso
\maketitle
\pagestyle{titlingpage}
\thispagestyle{titlingpage} % \pagestyle does not work on the first one…

\cleartoverso
{\small

  Copyright 2017 by Allex Oliveira.

  The contents of this book are protected under the Creative Commons
  Attribution-ShareAlike 3.0 Unported license.

  You are \textbf{free}:
  \begin{itemize}
  \item to \textbf{Share}: to copy, distribute and transmit the work,
  \item to \textbf{Remix}: to adapt the work,
  \end{itemize}

  Under the following conditions:
  \begin{description}
  \item[Attribution.] You must attribute the work in the manner specified by the
    author or licensor (but not in any way that suggests that they endorse you
    or your use of the work).
  \item[Share Alike.] If you alter, transform, or build upon this work, you may
    distribute the resulting work only under the same, similar or a compatible
    license.
  \end{description}

  For any reuse or distribution, you must make clear to others the
  license terms of this work. The best way to do this is with a link to
  this web page: \\
  \url{http://creativecommons.org/licenses/by-sa/3.0/}

  Any of the above conditions can be waived if you get permission from
  the copyright holder. Nothing in this license impairs or restricts the
  author's moral rights.

  \begin{center}
    \includegraphics[width=0.2\textwidth]{_support/latex/sbabook/CreativeCommons-BY-SA.pdf}
  \end{center}

  Your fair dealing and other rights are in no way affected by the
  above. This is a human-readable summary of the Legal Code (the full
  license): \\
  \url{http://creativecommons.org/licenses/by-sa/3.0/legalcode}

  \vfill

  % Publication info would go here (publisher, ISBN, cover design…)
  Layout and typography based on the \textcode{sbabook} \LaTeX{} class by Damien
  Pollet.
}


\frontmatter
\pagestyle{plain}

\tableofcontents*
\clearpage\listoffigures

\mainmatter

\chapter{Lesson 12 - Building a Webserver on Raspberry}
In the previous lessons, we learned how to control LEDs, sensors, LCD displays and how to use OOP to create applications to control them and how to build a Mini-Weather Station. Now we going to build a Webserver to interact with GPIOs.
\section{What do we need?}\subsection{Components}
\begin{itemize}
\item 1 Raspberry Pi connected to your network (wired or wireless)
\item Jumper wires
\end{itemize}
\section{Experimental theory}
Before constructing any circuit, you must know the parameters of the components in the circuit, such as their operating voltage, operating circuit, etc.
\section{Experimental procedure}\section{HTML page}
\begin{displaycode}{plain}
<html>
   <head>
      <title>Remote control</title>
      <!-- Bootstrap core CSS -->
      <link href="https://getbootstrap.com/docs/4.0/dist/css/bootstrap.min.css" rel="stylesheet">
   </head>
   <body >
      <main role="main">
         <section class="jumbotron text-center">
            <div class="container">
               <h1 class="jumbotron-heading">Remote control</h1>
               <p class="lead text-muted">Temperature: 28&deg;C</p>
               <p class="lead text-muted">Humidity: 42&#37;</p>
               <p class="lead text-muted">Pressure: 1017 hPa</p>
               <p class="lead text-muted">Fan state: 
                  <button type="button" class="btn btn-success" disabled="disabled">ON</button>
               </p>
               <p>
                  <a href="#" class="btn btn-primary my-2">Turn ON</a>
                  <a href="#" class="btn btn-secondary my-2">Turn OFF</a>
               </p>
            </div>
         </section>
      </main>
      <!-- Bootstrap core JavaScript -->
      <script src="https://code.jquery.com/jquery-3.2.1.slim.min.js" integrity="sha384-KJ3o2DKtIkvYIK3UENzmM7KCkRr/rE9/Qpg6aAZGJwFDMVNA/GpGFF93hXpG5KkN" crossorigin="anonymous"></script>
   </body>
</html>
\end{displaycode}

This page looks like the Picture \ref{WebPage}. 


\begin{figure}

\begin{center}
\includegraphics[width=0.85\textwidth]{/Users/allexoliveira/PharoThingsBook/Booklet-APharoThingTutorial/_result/pdf/Chapters/Chap13BuildingaWebServerOnRaspberry/figures/pharothings-webpage.png}\caption{Web Page.\label{WebPage}}\end{center}
\end{figure}



% lulu requires an empty page at the end. That's why I'm using
% \backmatter here.
\backmatter

% Index would go here

\end{document}
